\documentclass{acm}

\usepackage{fontawesome}
\usepackage{etoolbox}
\usepackage{textcomp}
\usepackage[nodisplayskipstretch]{setspace}
\usepackage{xspace}
\usepackage{verbatim}
\usepackage{multicol}
\usepackage{soul}
\usepackage{attrib}

\usepackage{amsmath,amssymb,amsthm}

\usepackage[linesnumbered,commentsnumbered,ruled,vlined]{algorithm2e}
\newcommand\mycommfont[1]{\footnotesize\ttfamily\textcolor{blue}{#1}}
\SetCommentSty{mycommfont}
\SetKwComment{tcc}{ \# }{}
\SetKwComment{tcp}{ \# }{}

\usepackage{siunitx}

\usepackage{tikz}
\usepackage{pgfplots}
\usetikzlibrary{decorations.pathreplacing,calc,arrows.meta,shapes,graphs}

\AtBeginEnvironment{minted}{\singlespacing\fontsize{10}{10}\selectfont}
\setmainfont{Open Sans Light}
\usefonttheme{serif}

\makeatletter
\patchcmd{\beamer@sectionintoc}{\vskip1.5em}{\vskip0.5em}{}{}
\makeatother

% Math stuffs
\newcommand{\Z}{\mathbb{Z}}
\newcommand{\R}{\mathbb{R}}
\newcommand{\N}{\mathbb{N}}
\newcommand{\lcm}{\text{lcm}}
\newcommand{\Inn}{\text{Inn}}
\newcommand{\Aut}{\text{Aut}}
\newcommand{\Ker}{\text{Ker}\ }
\newcommand{\la}{\langle}
\newcommand{\ra}{\rangle}

\newcommand{\yournewcommand}[2]{Something #1, and #2}

\newenvironment{question}[1]{\par\textbf{Question #1.}\par}{}

\newcommand{\pmidg}[1]{\parbox{\widthof{#1}}{#1}}
\newcommand{\splitslide}[4]{
    \noindent
    \begin{minipage}{#1 \textwidth - #2 }
        #3
    \end{minipage}%
    \hspace{ \dimexpr #2 * 2 \relax }%
    \begin{minipage}{\textwidth - #1 \textwidth - #2 }
        #4
    \end{minipage}
}

\newcommand{\frameoutput}[1]{\frame{\colorbox{white}{#1}}}

\newcommand{\tikzmark}[1]{%
\tikz[baseline=-0.55ex,overlay,remember picture] \node[inner sep=0pt,] (#1)
{\vphantom{T}};
}

\newcommand{\braced}[3]{%
    \begin{tikzpicture}[overlay,remember picture]
        \draw [thick,decorate,decoration={brace,raise=1ex,amplitude=4pt},blue] (#2.south west-|T1.south west) -- node[anchor=west,left,xshift=-1.8ex,text=olive]{#3} (#1.north west-|T1.south west);
    \end{tikzpicture}
}

\title{Maubot Workshop}
\author{Sumner Evans}
\institute{Beeper}
\date{March 1, 2022}

\begin{document}

\begin{frame}{What is Maubot?}
    Maubot is a \textbf{plugin-based} Matrix bot system built on top of the
    \textbf{Mautrix} library. Both libraries were created by Tulir (one of my
    coworkers).

    \pause
    Because it's plugin-based, you run a single maubot instance, and then use an
    \textit{admin panel} to upload and instantiate plugins.

    \pause
    It is built on top of mautrix-python, which is a powerful Python Matrix
    library that underpins all of the mautrix bridges including the Telegram,
    Facebook, and Instagram bridges.
\end{frame}

\begin{frame}{Maubot Admin Panel}
    I have set up a maubot instance here:

    \begin{center}
        \Large
        \url{https://argon.ohea.xyz/_matrix/maubot/}
    \end{center}

    Username: \texttt{demo} \\
    Password: \texttt{matrixiscool}
\end{frame}

\begin{frame}[fragile]{General bot-writing workflow}
    \textbf{Documentation:}
    \url{https://docs.mau.fi/maubot/dev/getting-started.html}

    \bigskip

    \textbf{Install maubot} on your local machine:
    \begin{minted}{shell}
    pip3 install --user maubot
    \end{minted}

    \textbf{Writing your bot:}
    \begin{itemize}
        \item Run \texttt{mbc init} for interactive plugin creation.
        \item Run \texttt{mbc login} to connect to the maubot instance.
        \item After you write your code, upload it to the server with: \\
            \texttt{mbc build --upload}
        \item Create a user for a client: \\
            \texttt{mbc auth --register --homeserver argon.ohea.xyz --server argon}
        \item Create a client and instance on the Maubot Manager.
    \end{itemize}
\end{frame}

\begin{frame}[standout]
    \Huge
    Example: Echobot
\end{frame}

\begin{frame}[standout]
    \Huge
    Example: Firefighter bot
\end{frame}

\begin{frame}{Bot Ideas}
    \textbf{Now it's your turn!} We recommend you start with something small (an
    existing bot for example), then build up from there.

    \begin{itemize}
        \item Try writing a cooler echobot or a better reaction bot.
        \item A bot that sends back ASCII-art of the given word or phrase.
        \item A bot that applies a filter to images sent in a room.
        \item A 20 questions bot.
        \item A dice roll bot
        \item Welcome bot that says hi to new users.
        \item Upvote/downvote tracker bot that tracks the number of thumbs up
            and thumbs down emojis that each user gets.
        \item Any other ideas?
    \end{itemize}
\end{frame}

\end{document}
% Local Variables:
% TeX-command-extra-options: "-shell-escape"
% End:
